\section{Glossary}
The words and abbreviations listed below when used in route instructions have only the following definitions when they appear in upper case (all capital letters) and not in quotation marks (``'').  All words in route instructions appearing in lower case are understood to mean the common dictionary definition.

\subsection{AFTER}
The first intersection beyond the referenced sign or landmark at which the instruction could be executed.

\subsection{API}
After previous NRI; at the indicated point beyond completion of the previous NRI.

\subsection{BEFORE}
The last intersection prior to and in sight of the referenced sign or landmark at which the instruction could be executed.

\subsection{BLINKER}
An intersection controlled by a conventional traffic caution or stop blinker(s), capable of operating as blinker(s) only.  May be off.

\subsection{CAST}
Change average speed to.

\subsection{CROSSROAD}
A crossing of two legal rally roads.  A CROSSROAD is an intersection and the roads may cross at any angle.

\subsection{CSD}
Change average speed down by the amount indicated.

\subsection{CSU}
Change average speed up by the amount indicated.

\subsection{DIYC}
Do It Yourself Checkpoint.  Upon encountering the DIYC reference, record your exact arrival time to the hundredth of a minute as your End Leg time on your scorecard.  (Use the seconds to hundredths conversion table in appendix \ref{apdx:sec-to-hdth-conversion} on page \pageref{apdx:sec-to-hdth-conversion}.)  Record your next Start Leg time as exactly two (2.00) minutes after your End leg time.  Use your timepiece to time yourself out from the DIYC checkpoint.

\subsection{FOLLOW}
See Section \ref{sec:follow} on page \pageref{sec:follow}.

\subsection{GAIN}
To make up a specified time during a specified or implied distance.  The distance over which a GAIN is operative is free of checkpoints and route controls.  The GAIN time is subtracted in the leg time calculations.

\subsection{ITIS}
If there is such.  Execute an ITIS instruction only if you encounter its action point before you come to the action point of the next (by number) NRI.  Otherwise, skip the ITIS instruction and consider it completed.

\subsection{L, LEFT}
Leftmost deviation of any angle off the main road.

\subsection{MBCU}
May be considered unnecessary.  A deviation labeled MBCU may be executed to follow the main road.

\subsection{OBSERVE}
To visually note and pass a sign or landmark.

\subsection{ONTO}
See Section \ref{sec:follow} on page \pageref{sec:follow}.

\subsection{OPP}
Opportunity.  A possible deviation in the direction indicated, that is paved and/or is named or numbered as indicated by sign(s) at the intersection.

\subsection{OR}
Complete an OR instruction by executing one-half of the given instruction, but not both, separated by the term OR.  The reference point of the two possibilities which is located first determines which half of the OR instruction is to be executed.  If both possibilities of an OR instruction have the same reference point, execution is determined by the action point which is located first.

\subsection{PAUSE}
To pause for a specified time.  The PAUSE time is added in the leg time calculations.

\subsection{R, RIGHT}
Rightmost deviation of any angle off the main road.

\subsection{S, STRAIGHT}
The straightest deviation off the main road within 45 degrees of straight ahead.

\subsection{SIDEROAD}
An intersection at which the contestant can TURN in only one direction.

\subsection{SIGNAL}
An intersection controlled by multi-light traffic signal(s) which may be operating as blinker(s) or may be off.

\subsection{SOL}
Sign must be on left.  May not always be given.

\subsection{STOP}
An intersection with an official highway stop sign at which the contestant is legally required to stop.

\subsection{T}
An intersection having the shape of the letter \textbf{T} as approached from the base.  It is not possible to execute the instruction S or the instruction STRAIGHT at a T.

\subsection{TOWARD}
See Section \ref{sec:follow} on page \pageref{sec:follow}.

\subsection{TRANSIT}
A part of the rally in which no checkpoints or route controls are located and for which there is no stated average speed.  Route instructions apply, but you may leave the route for rest or refueling stops.  The TRANSIT time is added in the leg time calculations.

\subsection{TURN}
A deviation from the main road in the only direction possible.  A TURN instruction cannot be executed if an instruction to go straight (S, STRAIGHT) would take the contestant on the same route.

\subsection{Y}
An intersection having the shape of the letter \textbf{Y} as approached from the base.  It is not possible to execute either the instruction S or the instruction STRAIGHT at a Y.

\subsection{YIELD}
An intersection with an official highway yield sign at which the contestant is legally required to yield.